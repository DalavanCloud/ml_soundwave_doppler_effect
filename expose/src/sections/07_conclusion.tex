\section{Fazit und Ausblick} 

In dieser Ausarbeitung wurden fünf verschiedene Verfahren vorgestellt eine
Gestenerkennung mit Hilfe des Doppler Effekts vorzunehmen. Dazu bedarf es eines
Klassifizeirungsalgorithmus, welcher die Rohdaten einer Geste zuordnen kann. Die
fünf Verschiedenen Verfahren \ac{SVM}, \acsu{HMM}, k-Means, Entscheidungsbäume und
\ac{LSTM} unterscheiden sich nicht nur im zugrunde liegenden Verfahren, auch in
den Umsetzungen gibt es erhebliche Unterschiede vor allem in der
Datenvorverarbeiteung. 

Die Gestenerkennung kann durch die verschiedenen Verfahren mit Einschränkungen
gut durchgeführt werden. Allerdings ist diese nicht universal einsetzbar. Es ist
aufgefallen, dass der Doppler Effekt sich bei unterschiedlicher
Hardwarekonfiguration unterschiedlich auswirkt. Dabei ist neben der Qualität der
Komponenten Mikrofon und Lautsprecher die Andordnung dieser Komponenten im
Laptop entscheiden. Die Variationen in der Anordnung reichen von Mikrofon
unterhalb des Bildschirms zu oberhalb über Mikrofon an der Vorderseite des
Gehäuses bis zu Lautsprecher seitlich der Tastatur zu unterhalb des vorderen
Gehäuse in verschiednen Kombinationen. Werden Beispieldaten mit verschiedenen
Hardwarekonfigurationen trainiert verschlechtert sich das
Klassifikationsergebnis, da sich verschiedene Gesten in den verschiedenen
Konfigurazionen gleichen. Des Weiteren sind auch Abhängigkeiten von der
benutzten Hand (links oder rechts), dem Abstand der Hand zum Mikrofon und
Bewegungen im Hintergrund erkennbar geworden. Aus diesem Grund ist es notwendig
für die verschiedenen Konfigurationen verschiedene trainierte Klassifikatoren
zur Verarbeitung vorzubereiten und diese entsprechend einzusetzen. Ein weiteres
Hindernis ist, dass einige Gesten sich stark ähneln, z.B. \ac{RO} mit \ac{BNN}
und \ac{BNS} oder \ac{SPO} mit \ac{RLO}. Das führt dazu, dass die Unterschiede
nur bei korrekter Ausführung der Geste deutlich hervortreten.
Klassifikatoren, die stark auf eine Datenvorverarbeitung mit Merkmalsextraktion
setzen wie z.B. k-Means oder Entscheidungsbäume, stoßen hier auf ihre Grenzen.
Auf die spezielle Konfigurationen ausgerichtete Klassifikatoren liefern jedoch
gute Ergebnisse. Auch bei anderen Klassifikatoren wie \ac{HMM}, \ac{SVM} und
\ac{LSTM} ist eine spezielle Konfiguration von Vorteil. beim \ac{LSTM} ist im
Vergleich am wenigsten Vorverarbeitung notwendig, da der Klassifikator durch das
training die Unterschiede selbstständig erkennt und dies in seinen Parametern
modelliert. Nachteil dieser Strategie ist jedoch der mit Abstand höchste
Trainingsaufwand (von einigen Stunden bis mehrere Tage und potentiell noch
länger) und die größte benötigte Menge an Beispieldaten (mehr als 4000
Beispiele). \ac{HMM} und \ac{LSTM} eignen sich für die Verarbeitung der Rohdaten
als Sequenz gut, da sie über interne Zustände verfügen, die eine zeitliche
Abhängigkeit modellieren können. Im Gegensatz dazu stehen die anderen drei
Verfahren \ac{SVM}, k-Means und Entscheidungsbäume, die die zeitliche
Abhängigkeit durch geignete Datenverarbeitung ohne Informationsverlust nicht
benötigen.

Das zentrale Problem der Live Klassifikation lässt sich auf alle fünf
Klassifikatoren anwenden. Während des Trainings sind Start- und Ende einer Geste
in einem groben Rahmen bekannt. Bei der Live Klassifikation ist dies jedoch
nicht der Fall. Daher sind Verfahren entwickelt worden, die entweder nach Start-
und Ende von Gesten suchen und die so gefundenen Daten klassifieziern oder
Verfahren die eine statistisches Modell und fortlaufend alles klassifizeiren,
dabei ist das beste Ergebnis in einem Intervall die gefundene Geste.

\paragraph{Ausblick} 
Die entworfenen Klassifikatoren enthalten bereits betrachtliche Ergebnisse.
Dennoch besteht Verbesserungsbedarf. Dies zielt vor allem auf einer besseren
Datenvorverarbeitung im Trainingsstadium sowie einer besseren Segementierung in
der Live Klassifikation ab. Mit weiteren Beispieldaten können weitere Merkmale
herausgefunden werden, um die Klassifikationsrate zu erhöhen. Des Weiteren ist
eine systematische Testreihe der Parameter der einzelnen Klassifikatoren und
deren Bibliotheksimplementierungen vorzunehmen, um die optimalen Ergebnisse zu
erhalten. Ein weiteres Problemfeld, dass es weiter zu verbessern gilt, ist der
Umgang mit unterschiedlichen hardwarekonfigurationen. Eine Idee hierzu ist es
verschieden trainierte Klassifikatoren zu verwenden und einen Mehrheitsentscheid
herbeizuführen oder entsprechend der aktuellen Hardwarekonfiguration einen
speziellen Klassifkator nur zu laden. Ein zusätzlicher Punkt der im Projekt
bisher nur nebenläufig behandelt wurde, ist die Integration mit dem Hostsystem.
Die Gesten sollen einer bestimmten Aktion zugeordnet werden. Dazu werden bei
erkannten Gesten zuvor definierte Aktionen durchgeführt, wie z.B. das drücken
der Escape Taste. Hierbei ist es wichtig, dass die Gesten nicht zu häufig falsch
erkannt werden, keine Klassifikation ist vor einer falschen Kalssifikation
vorzuziehen, da diese sonst unerwartete Aktionen hervorruft. Auch ist es
notwendig Gesten nicht zu schnell in einem kleinen Zeitintervall zu erkennen, um
eventuelle Bewegungen in Vor- und Nachbereitung zur Geste nicht auch noch zu
klassifizieren.







