\section{Trees/Ensemble Learning}

Da die Lautstärke der ausgesendeten Frequenz unterschiedlich sein kann, ist es sinnvoll die Daten 
zunächst zu normalisieren, sodass die Amplitude des Referenztons einen konstanten Wert hat. Ebenso 
werden die Amplituden der übrigen Frequenzanzteilen relativ zur normalisierten Amplitude angepasst.

Je nach Geste ändert sich die Bandbreite des Referenztones.
Die einzelnen Gesten unterscheiden sich in der unterschiedlichen Bandbreite des Frequenzspektrums 
um den Referenzton. Um eine Geste zu erkennen, werden die Frequenzanteile in beide Richtungen 
untersucht, wobei nur Frequenzen mit einer Amplitude größer als 10% der Amplitude des Referenztones 
relevant sind. Nur wenn Breite des Frequenzspektrums des Referenzspektrums einen Grenzwert 
überschreitet, handelt es sich um eine Geste. In Ausnahmefällen ist der Ausschlag der Amplitude 
durch ein Minimum der Amplitude vom Referenzton separiert, sodass man zunächst davon ausgehen würde, 
dass keine Geste stattgefunden hat. Deshalb wird weiter nach einem zweiten Ausschlag gesucht. Dessen 
Amplitude muss mindestens 30% der Amplitude des Referenztones haben.

Es bietet sich an, die Komplexität der generierten Eingabedaten zu verringern, da eine Klassifizierung 
der Daten zu Overfitting führen kann. 
Das Problem ist, dass nicht nur ein Datensatz pro Geste existiert, sondern ein sich über die Zeit änderndes 
Frequenzbild auszuwerten ist. 
Die Idee ist nun, dass dieser zeitliche Verlauf in Form eines Vektors abgebildet werden kann. Das Ziel ist es,
mit diesem Vektor die Frequenzverschiebung abzubilden und darüber die Gesten zu klassifizieren. 

Starkt vereinfacht unterscheiden sich die Gesten darin, ob und auf welcher Seite des Referenztones eine 
Verschiebung des Referenztones stattfindet. 
Zudem wird unterschieden in welche Richtung sich die Verschiebung ausbreitet. Sie kann beispielsweise von 
der rechten zur linken, bzw. linken zur rechten Seite wandern, sowie von der Mitte beginnend in eine 
Richtung ausschlagen.
Diese Unterscheidungsmerkmale können sehr gut durch einen Entscheidungsbaum abgebildet werden. Dazu kann 
die Verschiebungsrichtung als Maß dienen, um die Eingabemenge zu unterteilen. 

Die Besonderheit von Ensemble Learning ist, dass mit Hilfe von vielen schwachen Klassifikatoren 
gute Klassifikationsergebnisse erziehlt werden können. Die Entscheidungen, die die verschiedenen 
Klassifikatoren treffen, sind dabei verschieden, was jedoch auch die Stärke dieser Methode ist, 
denn sie ergänzen sich gegenseitig. 





