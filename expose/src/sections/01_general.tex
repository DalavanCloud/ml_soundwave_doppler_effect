\section{Einleitung}
\label{sec:intro}
Eine Alternative zu den Standardeingabegeräten wie Maus und Tastatur zur Interaktion mit Computern sind Gesten. 
Von besonderem Interesse ist dabei das Erkennen von sogenannten \glqq In-Air\grqq -Gesten, die mit den Händen in der Luft ausgeführt werden. Der Unterschied zu Gesten, wie sie beispielsweise auf Touch-Displays von Smartphones für das Rotieren oder Skalieren von Bildern verwendet werden, besteht darin, dass die Interaktion kontaktfrei ist und durch die Bewegung der Hände im Raum wesentlich komplexere Gesten realisiert werden können.

Zwei bekannte Geräte zur In-Air-Gestenerkennung sind Leap Motion~\cite{LeapMotion} und Microsoft Kinect~\cite{Kinect}. Diese Geräte verwenden optische Verfahren zur Erkennung der Hände oder Finger. Ein Nachteil beider Verfahren ist jedoch, dass sie wegen der optischen Erkennung sehr abhängig von der Beleuchtung sind. Weiterhin ist die Gesten-Erkennung erst durch die Verwendung zusätzlicher Hardware möglich.

Ein alternativer Ansatz ist die Verwendung von Mikrofon und Lautsprecher zur akustischen Erkennung von Gesten. Ein Vorteil dieses Ansatzes gegenüber Leap Motion oder Microsoft Kinect ist, dass die meisten Computer und Laptops mit dieser Hardware bereits ausgestattet sind.
Das in \cite{Gupta2012} beschriebene Verfahren nutzt dazu den Doppler-Effekt, also die Frequenzverschiebungen eines konstanten Tons bei zeitlicher Veränderung der Entfernung zwischen der Schallquelle (Sender) und dem Empfänger. 
Über die Lautsprecher (Sender) wird ein konstanter, hochfrequenter Ton ausgegeben, der von dem Mikrofon (Empfänger) aufgenommen wird. 
Da sowohl Sender als auch Empfänger eine feste Position haben, kommt es durch das Bewegen der Hände zu Frequenzverschiebungen, die sich auf den Doppler-Effekt zurückführen lassen. 
Diese Frequenzverschiebungen können analysiert und Rückschlüsse auf die Richtung und die Geschwindigkeit von der Bewegung gezogen werden. 

%Ziel des Projektes ist es ein Programm zu entwickeln, was Gesten erkennt. Dazu
%wurde von Microsoft ein Verfahren entwickelt, welches anhand der Doppler
%Verschiebung, eines Soundsignals, Gesten erkennt. Dieses Verfahren wird in
%\cite{Gupta2012} beschrieben. Um die Gesten zu erkennen sollen verschiedene
%Machine Learning Verfahren verwendet werden.

Im Rahmen der Lehrveranstaltung Machine Learning an der Hochschule RheinMain im Studiengang Master Informatik soll eine Anwendung mit dem Ziel entwickelt werden, die akustische Erkennung und Klassifikation von Gesten auf Grundlage des Doppler-Effekts durch verschiedene Machine Learning Verfahren zu realisieren. 

Die Anwendung soll folgende Gesten unterscheiden können:
\begin{itemize}
	\item \texttt{\ac{RLO}}\\
	Das Bewegen der Hand von rechts nach links vor dem Bildschirm
	\item \texttt{\ac{TBO}}\\
	Das Bewegen der Hand von oben nach unten vor dem Bildschirm
	\item \texttt{\ac{OT}}\\
	Beide Hände entgegengesetzt auf den Bildschirm hin und weg bewegen
	\item \texttt{\ac{SPO}}\\
	Mit einer Hand eine kurze Bewegung in Richtung des Monitors und zurück
	\item \texttt{\ac{DPO}}\\
	Mit einer Hand zwei kurze aufeinander folgende Bewegungen in Richtung des Monitors und zurück
	\item \texttt{\ac{RO}}\\
	Rotation mit einer Hand
\end{itemize}

Zur Abgrenzung der Gesten zu einer Nicht-Geste werden zusätzlich noch die Hintergrundgeräusche als Klasse definiert. Dies ist einmal \texttt{\ac{BNS}}, diese Geste spiegelt die Referenzfreuqenz wieder, und zum Anderen \texttt{\ac{BNN}}, bei dieser Geste herrscht zusätzlich noch beliebiger Lärm im Raum. 

\subsection{Aufnahmeprogramm}
\label{sec:gestures_dataformat}

Zur Aufnahme der Gesten wurde die Python Bibliothek \texttt{pyAudio} verwendet.
Die Referenzfrequenz liegt bei 18,5 kHz, welche abgespielt wird und die
reflektierte, mdifizeirte Frequenz (vgl.
Doppler-Effekt) wird aufgenommen. Die Aufnahme beschränkt sich auf den
Frequenzbereich von 18 kHz bis 19 kHz. Jede dieser Aufnahmen besteht aus
32 Datensätzen á 64 Datenpunkten in einem Zeitraum von 500ms. 

\subsection{Signalerzeugung}
Die Erzeugung des Soundsignals wurde mithilfe der Python Library wavebender und
pyAudio erreicht. In Listing \ref{lst:gen_sound} ist ein Bespiel für die
Erzeugung eines Audiosignals gezeigt. Für die Gestenerkennung wird ein Signal
mit 18,5 kHz erzeugt.

\enlargethispage{\baselineskip}
\enlargethispage{\baselineskip}
\enlargethispage{\baselineskip}
\begin{lstlisting}[caption={Erzeugung eines Soundsignals},label={lst:gen_sound}]{lst:gen_sound} 
audioDev = pyaudio.PyAudio()
channels = ((wb.sine_wave(frequency, amplitude=amplitude, framerate=framerate),),)
samples = wb.compute_samples(channels, framerate * duration * 1)
audioStream = MyStream(self.audioDev.open(format=audioDev.get_format_from_width(2),
					 channels=1, rate=framerate, output=True))
wb.write_wavefile(audioStream, samples)
\end{lstlisting}

% \subsection{Gesten aufnehmen und Datenformat}

\nocite{Gupta2012}