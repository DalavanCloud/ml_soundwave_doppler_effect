\newpage 
\subsection{Anwendung auf das Projekt}
\begin{itemize}
\item{normierung}
\item{quantisierung}
\item{transformation}
\end{itemize}

% UND-Perzeptron

\subsubsection{Datenaufbereitung}
In Kapitel \ref{Kapitel 1 TODO} wurde die Aufnahme und die generelle Vorverarbeitung der Daten bereits erl�utert, bevor diese an die entsprechenden Klassifikatoren weitergeleitet werden.
Die Klassifikatoren erhalten dabei jeweils einzelne Frames mit bereits normalisierten 64 Werten, die sich innerhab des Frequenzbereichs um die Referenzfrequenz von $18500\text{kHz}$ befinden  \{Bild einf�gen\}. 
Da jeder Klassifikator aufgrund seiner Funktionsweise unterschiedlich mit den Eingabedaten arbeitet, ist eine entsprechende und auf den jeweiligen Klassifikator angepasste Datenaufbereitung notwendig.

Dieser Abschnitt befasst sich mit der spezifischen Datenaufbereitung f�r den Support Vector Machine Klassifikator.

[..]
Bei der Betrachtung der Trainingsdaten beziehungsweise -Gesten f�llt auf, dass die durch den Dopper-Effekt verursachten Frequenzverschiebungen in einem relativ kleinen Bereich um die Referenzfrequenz herum sattfinden.
Zudem wird offensichtlich, dass nur ein Bruchteil der urspr�nglich veranschlagten 32 Frames pro Geste relevante Informationen enthalten, da eine ausgef�hrte Geste meist zwischen 8 und 16 Frames lang ist beziehungsweise nur innerhalb dieser Frame-Anzahl Frequenzverschiebungen verursacht.
Um diesen Overhead an nicht nutzbaren Daten zu beseitigen, wird die zu betrachtende Frame-Anzahl bei der Live-Klassifikation auf 20 festgelegt.
Der verwendete Wertebereich eines Frames wird sowohl in der Trainingsphase, als auch in der Live-Klassifikation, von 64 auf 40 verkleinert \{Bild einf�gen\}. 

Da nach der Vorverarbeitung der Daten (\ref{Kapitel 1 TODO} trotz Subtraktion der Referenzfrequenz teilweise noch sehr feines Rauschen in den Daten vorhanden sein kann, werden Werte unterhalb eines festgelegten Schwellenwerts (Threshold) auf 0 gesetzt.
Ein Aufsummieren aller Frames mit anschlie�ender Normalisierung liefert ein ''Gesamtergebnis'' einer Geste, in der sich alle relevanten Merkmale befinden.

Durch die Reduzierung des Wertebereichs auf 40 Werte pro Frame und durch das Aufsummieren zu einer Einheitsgeste wird eine Problemdimensionsreduzierung beim Training einer SVM um den Faktor ~51 erzielt.
Eine weitere Reduzierung um den Faktor 2 ist m�glich, sofern nur jeder zweite Wert der Einheitsgeste f�r das Training verwendet wird.


\subsubsection{Anpassung des Klassifikators}
- Ergebnisorientoiert

\subsubsection{Implementierung}
Aufgrund des Einsatzes von verschiedenen Klassifikatoren ist ein gemeinsames Interface f�r alle Klassifikatoren vorhanden, welches von jedem Klassifikator -- allerdings unterschiedlich -- implementiert ist. 
Dabei handelt es sich um die abstrakte Klasse \textit{IClassifier}.

F�r die Implementierung des \ac{SVM}-Klassifikators innerhalb der Klasse SVM wird die Python-Bibliothek \cite{scikit-learn} verwendet. 
Die Bibliothek stellt neben einer Vielzahl von weiteren Modulen und Klassen unter anderem auch eine Implementierung einer SVM zur Verf�gung.
F�r grundlegende Matrizenberechnungen wird die Python-Bibliothekt \cite{NumPy} verwendet. \cite{NumPy} zeichnet sich vor allem durch effiziente Berechnungsoperationen und eine Vielzahl an n�tzlicher Funktionen aus.
Zus�tzlich findet das in der Python-Installation enthaltene \cite{subprocess}-Modul Einsatz, um mittels erkannter Gesten verschiedene Programme starten und beenden zu k�nnen.

Die Methode \textit{classify} f�hrt die entsprechenden Vorverarbeitungsschritte pro �bergebenem Frame aus und speichert diesen in einer globalen Liste \textit{self.datalist}. 
Mittels einer Abfrage, ob der Frame �berhaupt relevante Informationen besitzt, wird der Gestenanfang erkannt und der entsprechende Index der Liste \textit{self.datalist} gespeichert.
Sobald 20 weitere Frames in dieser Liste abgespeichert worden sind, startet zun�chst ein weiterer Vorverarbeitungsschritt, in dem alle Frames zu einer Einheitsgeste aufsummiert werden.
Die normalisierte Einheitsgeste ist das finale Ergebnis der Vorverarbeitung der Live-Klassifikation und dient dem SVM-Klassifikator als Eingabedatum.
Als R�ckgabewert liefert der Klassifikator die entsprechende Gestennummer, die er dem Eingabedatum zugeordnet hat.
Mittels dieser Nummer wird eine Methode aufgerufen, in der jeder Nummer eine entsprechende Funktion zugeordnet ist.

\begin{lstlisting}[language=Python,caption={Classify},label={lst:svm_classify}]{lst:svm_classify}
def classify(self, data):
	# init num samples per frame and borders
	sliced_num_samples_per_frame = SVM.NUM_SAMPLES_PER_FRAME - SVM.SLICE_RIGHT - SVM.SLICE_RIGHT
	left_border = SVM.SLICE_LEFT
	right_border = SVM.NUM_SAMPLES_PER_FRAME - SVM.SLICE_RIGHT

	normalized_data_with_noise = data / np.amax(data)
	normalized_data = normalized_data_with_noise[left_border:right_border] - self.noise_frame[
																			 left_border:right_border]

	frame = normalized_data
	irrelevant_samples = np.where(frame <= 0.025)
	frame[irrelevant_samples] = 0.0

	rr = 20  # whats that?!?
	self.datalist.append(frame)
	self.datanum += 1

	if np.amax(frame) > 0.0 and self.found == False:
		self.index = self.datanum
		self.found = True

	if self.index + rr == self.datanum and self.found == True:
		self.index = 0
		self.found = False

		current_frameset = np.asarray(self.datalist[-rr:])
	
		gesture_frame = current_frameset.sum(axis=0)

		try:
			normalised_gesture_frame = gesture_frame / np.amax(gesture_frame)
			if not np.isnan(np.sum(normalised_gesture_frame)):
				target_prediction = self.classifier.predict(normalised_gesture_frame[::2])[0]  # only each second?!?
				self.executeCommand(target_prediction)
		except:
			print "error =("

	if self.datanum > sliced_num_samples_per_frame:
		del self.datalist[0]
\end{lstlisting}


\begin{lstlisting}[language=Python,caption={Start Programms},label={lst:svm_startprogramms}]{lst:svm_startprogramms}
def executeCommand(self, number):
	print number,

	if number == 0 and self.executed["notepad"] == False:
		print "starting notepad"
		proc = sp.Popen("notepad")
		self.executed["notepad"] = proc.pid

	elif number == 1 and self.executed["notepad"] != False:
		sp.Popen("TASKKILL /F /PID {pid} /T".format(pid=self.executed["notepad"]))
		print "terminating notepad"
		self.executed["notepad"] = False

	elif number == 2 and self.executed["taskmgr"] == False:
		print "starting taskmanager"
		proc = sp.Popen("taskmgr")
		self.executed["taskmgr"] = proc.pid

	elif number == 3 and self.executed["taskmgr"] != False:
		sp.Popen("TASKKILL /F /PID {pid} /T".format(pid=self.executed["taskmgr"]))
		print "terminating taskmanager"
		self.executed["taskmgr"] = False

	elif number == 4 and self.executed["calc"] == False:
		print "starting calculator"
		proc = sp.Popen("calc")
		self.executed["calc"] = proc.pid

	elif number == 5 and self.executed["calc"] != False:
		sp.Popen("TASKKILL /F /PID {pid} /T".format(pid=self.executed["calc"]))
		print "terminating calculator"
		self.executed["calc"] = False

	elif number == 6:
		print "noise, do nothing ..."

	print ""
\end{lstlisting}


\subsubsection{Training}
- Trainingsmethode