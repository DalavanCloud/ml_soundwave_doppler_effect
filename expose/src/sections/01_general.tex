\section{Einleitung}
\label{sec:intro}
Eine Alternative zu den Standardeingabegeräten wie Maus und Tastatur zur
Interaktion mit Computern sind Gesten.
Von besonderem Interesse ist dabei das Erkennen von sogenannten \glqq
In-Air\grqq -Gesten, die mit den Händen in der Luft ausgeführt werden. Der
Unterschied zu Gesten, wie sie beispielsweise auf Touch-Displays von Smartphones
für das Rotieren oder Skalieren von Bildern verwendet werden, besteht darin,
dass die Interaktion kontaktfrei ist und durch die Bewegung der Hände im Raum
wesentlich komplexere Gesten realisiert werden können.

Zwei bekannte Geräte zur In-Air-Gestenerkennung sind \textit{Leap
Motion}~\cite{LeapMotion} und \textit{Microsoft Kinect}~\cite{Kinect}. Diese Geräte
verwenden optische Verfahren zur Erkennung der Hände oder Finger. Ein Nachteil
beider Verfahren ist jedoch, dass sie wegen der optischen Erkennung sehr
abhängig von der Beleuchtung sind. Weiterhin ist die Gesten-Erkennung erst durch
die Verwendung zusätzlicher Hardware möglich.

Ein alternativer Ansatz ist die Verwendung von Mikrofon und Lautsprecher zur
akustischen Erkennung von Gesten. Ein Vorteil dieses Ansatzes gegenüber Leap
Motion oder Microsoft Kinect ist, dass die meisten Computer und Laptops mit
dieser Hardware bereits ausgestattet sind.
Das in \cite{Gupta2012} beschriebene Verfahren nutzt dazu den Doppler-Effekt,
also die Frequenzverschiebungen eines konstanten Tons bei zeitlicher Veränderung
der Entfernung zwischen der Schallquelle (Sender) und dem Empfänger.
Über die Lautsprecher (Sender) wird ein konstanter, hochfrequenter Ton
ausgegeben, der von dem Mikrofon (Empfänger) aufgenommen wird.
Da sowohl Sender als auch Empfänger eine feste Position haben, kommt es durch
das Bewegen der Hände zu Frequenzverschiebungen, die sich auf den Doppler-Effekt
zurückführen lassen. 
Diese Frequenzverschiebungen können analysiert und es können Rückschlüsse auf die Richtung
und die Geschwindigkeit von der Bewegung gezogen werden.

% Ziel des Projektes ist es ein Programm zu entwickeln, was Gesten erkennt. Dazu
% wurde von Microsoft ein Verfahren entwickelt, welches anhand der Doppler
% Verschiebung, eines Soundsignals, Gesten erkennt. Dieses Verfahren wird in
% \cite{Gupta2012} beschrieben. Um die Gesten zu erkennen sollen verschiedene
% Machine Learning Verfahren verwendet werden.

Im Rahmen der Lehrveranstaltung Machine Learning an der Hochschule RheinMain im
Studiengang Master Informatik soll eine Anwendung mit dem Ziel entwickelt
werden, die akustische Erkennung und Klassifikation von Gesten auf Grundlage des
Doppler-Effekts durch verschiedene Machine Learning Verfahren zu realisieren.

Die Anwendung soll folgende Gesten unterscheiden können:
\begin{itemize}
	\item \texttt{\ac{RLO}}\\
	Das Bewegen der Hand von rechts nach links vor dem Bildschirm
	\item \texttt{\ac{TBO}}\\
	Das Bewegen der Hand von oben nach unten vor dem Bildschirm
	\item \texttt{\ac{OT}}\\
	Beide Hände entgegengesetzt auf den Bildschirm hin und wegbewegen
	\item \texttt{\ac{SPO}}\\
	Mit einer Hand eine kurze Bewegung in Richtung des Monitors und zurück
	\item \texttt{\ac{DPO}}\\
	Mit einer Hand zwei kurze aufeinander folgende Bewegungen in Richtung des Monitors und zurück
	\item \texttt{\ac{RO}}\\
	Rotation mit einer Hand
\end{itemize}

Zur Abgrenzung der Gesten zu einer Nicht-Geste werden zusätzlich noch die
Hintergrundgeräusche als Klasse definiert. Dies ist einmal \texttt{\ac{BNS}},
diese Geste spiegelt die Referenzfrequenz wieder, und zum Anderen
\texttt{\ac{BNN}}, bei dieser Geste herrscht zusätzlich noch beliebiger Lärm im
Raum.

\subsection{Klassifikationsverfahren}
Die Gestenerkennung soll anhand verschiedener Klassifikationsverfahren erfolgen.
Dabei müssen die Rohdaten unterschiedlich verarbeitet und auf verschiedene
Methoden angewendet werden, um eine Klassifikation zu erreichen. Es werden die
Vor- und Nachteile der unterschiedlichen Verfahren auf das Projekt bezogen
diskutiert. Jedes Klassifikationsverfahren ist in einem Kapitel im Detail
beschrieben. Folgende fünf Verfahren werden in den Kapiteln untersucht:
\begin{flushleft}
\begin{itemize}
	\item \ref{mainsec:svm}. \nameref{mainsec:svm}
	\item \ref{mainsec:hmm}. \nameref{mainsec:hmm}
	\item \ref{mainsec:kmeans}. \nameref{mainsec:kmeans}
	\item \ref{mainsec:trees}. \nameref{mainsec:trees}
	\item \ref{mainsec:lstm}. \nameref{mainsec:lstm}
\end{itemize}
\end{flushleft}

\subsection{Rahmenprogramm}
\label{sec:app}
Alle fünf Klassifikatoren nutzen ein gemeinsames Grundprogramm, welches die
gemeinsamen Aufgaben bewältigt. Das Programm ist in \texttt{Python} geschrieben
und ist für die Tonerzeugung und -aufnahme sowie für die allgemeine Steuerung
verantwortlich. 

Die Referenzfrequenz von 18,5 kHz wird kontinuierlich im Modul
\texttt{soundplayer.py} abgespielt. Mit dem Modul \texttt{recorder.py} werden
die Audiosignale mit einer Samplingrate von 48000 aufgenommen und mit Hilfe einer
Fouriertransformation zu einem Datenobjekt mit dem Frequenzspektrum im Bereich
von ~18 kHz bis ~19 kHz umgewandelt. Die so aufgenommenen Rohdaten werden
entweder bei der Aufnahme von Beispielen abgespeichert oder bei der Live-Klassifikation dem genutzten Klassifikator übergeben. Das aufgenommene Array des
Frequenzspektrums enthält 64 Werte. Dieses Array wird als \textit{Frame}
bezeichnet.

Des Weiteren wird das einheitliche Interface \texttt{IClassifier} für die
Klassifikatoren verwendet, um die Steuerung der einzelnen Klassifikatoren zu
vereinheitlichen. Diese können über eine integrierte Konsole angesteuert werden.
In der Konsole können über verschiedene Befehle die Live-Klassifizierung, das
Training, eine Validierung etc. gestartet werden.
\enlargethispage{\baselineskip}
Die Konsole bietet auch die Möglichkeit, Beispieldaten aufzunehmen. Eine
Beispielgeste besteht aus 32 aufeinanderfolgenden Frames. Die Aufnahme kann auch
im Batchverfahren direkt hintereinander erfolgen (mehrmals die gleiche Geste
ohne zusätzliche Interaktion). Aufgenommene Gesten werden pro Nutzer in
einzelnen Gesten-Ordnern gespeichert. Das Rahmenprogramm bietet auch eine
Schnittstelle zum Laden der Beispieldaten in ein NumPy Array. Dies wird im Modul
\texttt{gesturefileio.py} vorgenommen. Dabei können verschiedene Datenformate
geladen und es kann auch schon eine einfache Normalisierung vorgenommen werden.


