\section{\acl{HMM}}
\textit{Paul Pasler, Sebastian Rieder}

\acl{HMM} sind prädestiniert für diese Aufgabe \cite{mmmFink}, leider sind die APIS scheiße bzw wir zu blöd\ldots
\cite{rabiner}
Grundlage des \acl{HMM} waren die nach dem russischen Mathematiker Andrej Andrejewitsch Markov 
(1856 - 1922, siehe \cite{wiki:markov}) benannten Markov-Modelle. Zu Beginn des 20. Jahrhunderts 
beschäftigte er sich als erster mit einer statistischen Beschreibung von Zustands- und Symbolfolgen. 
Er führte eine statistische Analyse der Buchstabenfolge des Textes ``Eugen Onegin'' von Alexander 
Pushkin.

Der amerikanischen Mathematiker Leonard E. Baum (* 1931) und andere autoren beschäftigten sich Ende der 
sechziger Jahre erstmals mit \acl{HMM}. Erste Programme wurden zur Spracherkennung und später auch in der Bioinformatik  
 

\subsection{Funktionsweise}
\acl{HMM} erweitern das Konzept der Markov-Ketten um eine zustandsspezifische Ausgabe und eine statistisch 
modellierte Zustandsfolge.


\ubsection
