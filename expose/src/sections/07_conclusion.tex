\section{Fazit und Ausblick} 

In dieser Ausarbeitung wurden fünf verschiedene Verfahren vorgestellt, um eine
Gestenerkennung mit Hilfe des Doppler Effekts vorzunehmen. Dazu benötigt man zum
einen eine Vorverarbeitung der Daten, damit Merkmale aus den Rohdaten
herausgefiltert werden können und zum anderen einen Klassifikator, der die Daten
einer bestimmten Geste zuordnet. Die fünf verschiedenen Verfahren \ac{SVM},
\acsu{HMM}, k-Means, Entscheidungsbäume und \ac{LSTM} unterscheiden sich nicht
nur in den zugrundeliegenden Verfahren, sondern auch in der Umsetzungen gibt es
erhebliche Unterschiede, vor allem in der Datenvorverarbeitung.

Die Gestenerkennung kann durch die verschiedenen Verfahren mit Einschränkungen
gut durchgeführt werden. Allerdings ist diese nicht universal einsetzbar. Es ist
aufgefallen, dass der Doppler Effekt sich bei unterschiedlicher
Hardwarekonfiguration unterschiedlich auswirkt. Dabei ist neben der Qualität der
Komponenten Mikrofon und Lautsprecher die Anordnung dieser Komponenten
entscheidend. Grundsätzliche Unterschiede bestehen darin, dass sich
beispielsweise bei Notebooks Mikrofon und Lautsprecher an verschiedenen Stellen
befinden und so in unterschiedlicher Weise zusammenarbeiten. Eine Bewegung auf
das Display zu kann dann je nach Anordnung die Geste \acl{SPO} oder \acl{TBO}
sein. Werden Beispieldaten mit verschiedenen Hardwarekonfigurationen trainiert,
verschlechtert sich das Klassifikationsergebnis. Des Weiteren sind auch
Abhängigkeiten von der benutzten Hand (links oder rechts), dem Abstand der Hand
zum Mikrofon und Bewegungen im Hintergrund erkennbar geworden. Aus diesem Grund
ist es notwendig, für die verschiedenen Konfigurationen verschiedene trainierte
Klassifikatoren zur Verarbeitung vorzubereiten und diese entsprechend
einzusetzen. Ein weiteres Hindernis ist, dass einige Gesten sich stark ähneln,
z.B. \ac{RO} mit \ac{BNN} und \ac{BNS} oder \ac{SPO} mit \ac{RLO}. Das führt
dazu, dass die Unterschiede nur bei korrekter Ausführung der Geste deutlich
hervortreten.
Klassifikatoren, die stark auf eine Datenvorverarbeitung mit Merkmalsextraktion
setzen wie z.B. k-Means oder Entscheidungsbäume, stoßen hier an ihre Grenzen.
Auf die spezielle Konfigurationen ausgerichtete Klassifikatoren liefern jedoch
gute Ergebnisse. Auch bei anderen Klassifikatoren wie \ac{HMM}, \ac{SVM} und
\ac{LSTM} ist eine spezielle Konfiguration von Vorteil. Bei \ac{LSTM} ist im
Vergleich am wenigsten Vorverarbeitung notwendig, da der Klassifikator durch das
Training die Unterschiede selbstständig erkennt und dies in seinen Parametern
modelliert. Nachteil dieser Strategie ist jedoch der mit Abstand höchste
Trainingsaufwand (von einigen Stunden bis mehrere Tage und potentiell noch
länger) und die größte benötigte Menge an Beispieldaten (mehr als 4000
Beispiele). \ac{HMM} und \ac{LSTM} eignen sich für die Verarbeitung der Rohdaten
als Sequenz gut, da sie über interne Zustände verfügen, die eine zeitliche
Abhängigkeit modellieren können. Im Gegensatz dazu stehen die anderen drei
Verfahren \ac{SVM}, k-Means und Entscheidungsbäume, die die zeitliche
Abhängigkeit durch geeignete Datenverarbeitung ohne Informationsverlust nicht
benötigen.

Das zentrale Problem der Live Klassifikation lässt sich auf alle fünf
Klassifikatoren anwenden. Während des Trainings sind Start und Ende einer Geste
in einem groben Rahmen bekannt. Bei der Live Klassifikation ist dies jedoch
nicht der Fall. Daher sind Verfahren entwickelt worden, die entweder nach Start
und Ende von Gesten suchen und die so gefundenen Daten klassifizieren oder
Verfahren, die fortlaufend alles klassifizieren,
dabei ist das häufigste Ergebnis in einem Intervall die gefundene Geste.

\paragraph{Ausblick} 
Die entworfenen Klassifikatoren erzeugen bereits beträchtliche Ergebnisse.
Dennoch besteht Verbesserungsbedarf. Dies zielt vor allem auf eine bessere
Datenvorverarbeitung im Trainingsstadium sowie einer besseren Segmentierung in
der Live Klassifikation ab. Mit weiteren Beispieldaten können zusätzliche Merkmale
herausgefunden werden, um die Klassifikationsrate zu erhöhen. Des Weiteren ist
eine systematische Testreihe der Parameter der einzelnen Klassifikatoren und
deren Bibliotheksimplementierungen vorzunehmen, um die optimalen Ergebnisse zu
erhalten. Ein weiteres Problemfeld, das es weiter zu verbessern gilt, ist der
Umgang mit unterschiedlichen Hardwarekonfigurationen. Eine Idee hierzu ist es,
verschieden trainierte Klassifikatoren zu verwenden und einen Mehrheitsentscheid
herbeizuführen oder entsprechend der aktuellen Hardwarekonfiguration nur einen
speziellen Klassifikator zu laden. 

Ein zusätzlicher Punkt, der im Projekt bisher nur am Rande behandelt wurde, ist,
die Gesten für eine tatsächliche Anwendung zu verwenden, sodass man damit
beispielsweise Programme sinnvoll steuern kann. Für eine Nutzung sollte
allerdings die Klassifikation so fehlerfrei wie möglich funktionieren, um
Fehlfunktionen zu vermeiden.


% Ein zusätzlicher Punkt der im Projekt bisher nur am Rande behandelt wurde, ist
% die Integration mit dem Hostsystem. Die Gesten sollen einer bestimmten Aktion
% zugeordnet werden. Dazu werden bei erkannten Gesten zuvor definierte Aktionen
% durchgeführt, wie z.B. das Drücken der Escape Taste. Hierbei ist es wichtig,
% dass die Gesten nicht zu häufig falsch erkannt werden, keine Klassifikation
% ist vor einer falschen Klassifikation vorzuziehen, da diese sonst unerwartete
% Aktionen hervorruft. Auch ist es notwendig Gesten nicht zu schnell in einem
% kleinen Zeitintervall zu erkennen, um eventuelle Bewegungen in Vor- und
% Nachbereitung zur Geste nicht auch noch zu klassifizieren.







