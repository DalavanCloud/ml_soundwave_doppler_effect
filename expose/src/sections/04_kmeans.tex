\section{k-Means}
\textit{Alex Baumgärtner, Robert Brylka}
\subsection{Kurzbeschreibung des Klassifikators}
Bei k-Means handelt es sich um einen Algorithmus für unüberwachtes Lernen, der eine Datenmenge in k verschiedene Cluster unterteilt und dabei jedes Objekt einem Cluster zuordnet. Dazu existieren k Cluster-Schwerpunkte (cluster centers), wobei ein Objekt immer dem nächstgelegenen Cluster-Schwerpunkt zugeordnet wird.

Zu Beginn der Lernphase des Algorithmus wird ein Wert für k festgelegt und anschließend werden k zufällige Positionen im Eingaberaum als Cluster-Schwerpunkte gesetzt. Danach wird jeder Schwerpunkt so verschoben, dass der Abstand zwischen ihm und den zugeordneten Objekten einen minimalen Wert annimmt. Diese Verschiebungen werden solange wiederholt, bis sich die Cluster-Schwerpunkte nicht mehr bewegen.

Um nun zu einem Objekt den passenden Cluster zuzuordnen, wird der Abstand zwischen Objekt und den Cluster-Schwerpunkten berechnet. Das Objekt wird dann dem Cluster-Schwerpunkt mit dem geringsten Abstand zugeordnet.


\subsection{Verwendung des Klassifikators im Gestenerkennungssystem}
Bei dem k-Means Algorithmus ist zu beachten, dass die Anzahl der Cluster vorher bekannt sein muss, da sie beim Start des Algorithmus festgelegt wird. Dies ist jedoch kein Problem, weil die Anzahl der Gesten im Erkennungssystem auf eine bestimmte Zahl (voraussichtlich 6) festgelegt ist. Da der Klassifikator jede beliebige Eingabe einem der Cluster zuordnet, ist es notwendig, eine oder mehrere zusätzliche Gruppen für nicht erkannte Gesten zu definieren. Ansonsten würde jede Eingabe – unabhängig ob es tatsächlich eine Geste war oder nicht – grundsätzlich der ähnlichsten bekannten Gesten zugeordnet. Denkbar sind hierbei beispielsweise zwei solcher zusätzlicher Gruppen, eine für das Hintergrundrauschen ohne Bewegung und eine weitere für alle Bewegungen, die keiner gespeicherten Geste entsprechen.

Die Features zur Klassifikation müssen so gewählt werden, dass sich gleiche Gesten im gleichen Cluster anordnen und es möglichst nicht zu einer Überschneidung von verschiedenen Clustern kommt. Dies ist notwendig, da der Klassifikator ansonsten nicht mehr in der Lage ist, die einzelnen Gesten voneinander zu unterscheiden. Anhand des im aktuellen Prototyp dargestellten Frequenzspektrums lässt sich erkennen, dass sich die Gesten weniger durch die zu einem bestimmten Zeitpunkt gemessenen Frequenzwerte unterscheiden lassen, sondern eher durch den zeitlichen Verlauf dieser Frequenzwerte von Beginn bis Ende der Geste. Diese Beobachtung führt zu folgenden ersten Ideen, welche Daten des Frequenzspektrums zur Klassifikation verwendet werden können:

\begin{enumerate}
\item \emph{Peaks und deren zeitlicher Verlauf:} Zu jedem Zeitpunkt einer Messung werden die Frequenzen ermittelt, bei denen lokale Maxima (Peaks) auftreten. Hierbei ist zu überlegen, die Anzahl der Peaks auf einen vorher festgelegten Maximalwert zu begrenzen, sodass beim Überschreiten dieses Maximalwerts nur die Peaks mit der höchsten Lautstärke berücksichtigt werden.
\item \emph{Überschreitungen eines Schwellwerts und deren zeitlicher Verlauf:} Es wird für jeden von der Fourier-Transformation ausgewerteten Frequenzbereich überprüft, ob die zugehörige Lautstärke einen Schwellwert überschreitet. Falls ja wird diesem der Wert 1, ansonsten der Wert 0 zugeordnet. Ebenso ist eine Erweiterung dieses Verfahrens denkbar, bei dem mehrere solcher Schwellwerte existieren und dem Frequenzbereich der Wert 1 beim Überschreiten des niedrigsten Schwellwerts, der Wert 2 beim Überschreiten des nächsthöheren Schwellwerts usw. zugeordnet wird. 
\item\emph{ Kombination der in Punkt 1 und 2 bestimmten Daten.}
\end{enumerate}